%%%%%%%%%%%%%%%%%%%%%%%%%%%%%%%%%%%%%%%%%%%%
%                                          %
%    SKIP TO THE NEXT COMMENTS BELOW       %
%   AND FOLLOW THE INSTRUCTIONS THERE      %
%                                          %
%%%%%%%%%%%%%%%%%%%%%%%%%%%%%%%%%%%%%%%%%%%%

\documentclass[12pt]{amsart}
\usepackage{amssymb,ifthen}

\setlength{\oddsidemargin}{-0.0cm}
\setlength{\evensidemargin}{-0.0cm} \setlength{\textwidth}{15.5cm}
\setlength{\textheight}{23.7cm} 
\pagestyle{empty}
\parindent=6mm
\def\emptystring{}

\renewcommand{\abstract}[6]{\setcounter{equation}{0}
\hbox{}\begin{center}\vskip-2.4truecm
{\small\sc 数学と物理における Julia の活用}\\[1mm]
{\tiny \bf July 11th, 2023, Kyushu University, Fukuoka, Japan}
\end{center}\vspace{1mm}\hrule\vspace{1.2truecm}
\begin{center}
 {\Large\bf #4}\\\vspace{5mm}{\large\bf #2 #1}\\\vspace{2mm}{#3}\\
 \ifthenelse{\equal{#5}{\emptystring}}{}
 {\normalsize \vspace{2mm}(joint work with #5)}
\end{center}
\vspace{7mm} #6}

\begin{document}
\thispagestyle{empty}

%%%%%%%%%%%%%%%%%
%  \abstract is a command of 6 arguments that 
%  should be completed via the following sketch 
%  (the 5th argument should be left empty if you 
%   have no coauthor):
%                                                 
%   \abstract{Family name}{Given name(s)}{Institution}{Title of talk}{Coauthor(s)}{Text of Abstract}
%%%%%%%%%%%%%%%%%

%Example
\abstract{Furihata}{Daisuke}
{Osaka University, Japan}
{応用数学の概念を用いた物理モデルをいかに Julia でプログラミングを行うか}{}
{

物理上の相分離現象に対するモデル方程式としては Cahn-Hilliard 方程式が有名かつ有用だが,この数値解析は計算量的に高コストである.そこでこの現象のごく初期過程を除いて状況を記述するようなシンプルなモデルを考案したい. この際,流体的挙動を模するために空間の Voronoi 分割という応用数学的手法をモデルに取り入れるのだが,これまでのコンピュータ言語ではこうしたモデルの数値解析にはそれなりに苦労する面が多々あったのが現状である.これに対し,Julia を用いると比較的容易に数値解析が行えることを示すことで,微分方程式の数値解析に対して Julia がどのように用いられるかを紹介したい.



}
\end{document}
