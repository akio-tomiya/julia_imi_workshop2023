%%%%%%%%%%%%%%%%%%%%%%%%%%%%%%%%%%%%%%%%%%%%
%                                          %
%    SKIP TO THE NEXT COMMENTS BELOW       %
%   AND FOLLOW THE INSTRUCTIONS THERE      %
%                                          %
%%%%%%%%%%%%%%%%%%%%%%%%%%%%%%%%%%%%%%%%%%%%

\documentclass[12pt]{amsart}
\usepackage{amssymb,ifthen}

\setlength{\oddsidemargin}{-0.0cm}
\setlength{\evensidemargin}{-0.0cm} \setlength{\textwidth}{15.5cm}
\setlength{\textheight}{23.7cm} 
\pagestyle{empty}
\parindent=6mm
\def\emptystring{}

\renewcommand{\abstract}[6]{\setcounter{equation}{0}
\hbox{}\begin{center}\vskip-2.4truecm
{\small\sc 数学と物理における Julia の活用}\\[1mm]
{\tiny \bf July 11th, 2023, Kyushu University, Fukuoka, Japan}
\end{center}\vspace{1mm}\hrule\vspace{1.2truecm}
\begin{center}
 {\Large\bf #4}\\\vspace{5mm}{\large\bf #2 #1}\\\vspace{2mm}{#3}\\
 \ifthenelse{\equal{#5}{\emptystring}}{}
 {\normalsize \vspace{2mm}(joint work with #5)}
\end{center}
\vspace{7mm} #6}

\begin{document}
\thispagestyle{empty}

%%%%%%%%%%%%%%%%%
%  \abstract is a command of 6 arguments that 
%  should be completed via the following sketch 
%  (the 5th argument should be left empty if you 
%   have no coauthor):
%                                                 
%   \abstract{Family name}{Given name(s)}{Institution}{Title of talk}{Coauthor(s)}{Text of Abstract}
%%%%%%%%%%%%%%%%%

%Example
\abstract{Horikawa}{Yuto}
{Osaka University, Japan}
{BasicBSpline.jl で始める B-spline}{}
{

B-spline は滑らかさを保証した区分多項式を作るための道具であり,幾何形状表現や数値解析など広範な分野で活用されている.Julia における B-spline のパッケージは幾つか存在するが,その中でも BasicBSpline.jl を本講演では扱う.BasicBSpline.jl パッケージの使用方法を交えながら B-spline に関する理論を解説し,その応用例についても触れる.





}
\end{document}
